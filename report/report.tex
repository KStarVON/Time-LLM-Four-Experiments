\documentclass[conference]{IEEEtran}
\IEEEoverridecommandlockouts
% The preceding line is only needed to identify funding in the first footnote. If that is unneeded, please comment it out.
\usepackage{cite}
\usepackage{amsmath,amssymb,amsfonts}
\usepackage{algorithmic}
\usepackage{graphicx}
\usepackage{textcomp}
\usepackage{xcolor}
\usepackage{listings}


\begin{document}

\title{Time-LLM Experiments' Results and Analysis}

\maketitle



\section{Results}

\subsection{Long-term forecasting}

\begin{tabular}{|c|c|c|}
    \hline
                        & \multicolumn{2}{c|}{ETTh1 96}         \\
    \cline{2-3}
    Model               & MSE                           & MAE   \\
    \hline
    GPT2-small Time-LLM & 0.387                         & 0.412 \\
    \hline
    LLama7b Time-LLM    & 0.362                         & 0.392 \\
    GPT4TS              & 0.376                         & 0.397 \\
    DLinear             & 0.375                         & 0.399 \\
    PatchTST            & 0.370                         & 0.399 \\
    TimesNet            & 0.384                         & 0.402 \\
    FEDformer           & 0.376                         & 0.419 \\
    Autoformer          & 0.449                         & 0.459 \\
    Stationary          & 0.513                         & 0.491 \\
    ETSformer           & 0.494                         & 0.479 \\
    LightTS             & 0.424                         & 0.432 \\
    Informer            & 0.865                         & 0.713 \\
    Reformer            & 0.837                         & 0.728 \\
    N-BEATS             & 0.496                         & 0.475 \\
    N-HiTS              & 0.392                         & 0.407 \\
    AutoARIMA           & 0.933                         & 0.635 \\
    AutoTheta           & 1.266                         & 0.758 \\
    AutoETS             & 1.264                         & 0.756 \\
    \hline
\end{tabular}

\begin{enumerate}
    \item MSE (Mean Squared Error): It is the average of the squares of the errors—that is, the average squared difference between the estimated values and the actual value. The smaller the MSE, the higher the model's prediction accuracy.
          \[\text{MSE} = \frac{1}{H} \sum_{h=1}^{H} \left(Y_{h} - \hat{Y}_{h}\right)^2\]
    \item MAE (Mean Absolute Error): It is the average of the absolute differences between prediction and actual observation. Like MSE, the smaller the MAE, the higher the model's prediction accuracy.
          \[\text{MAE} = \frac{1}{H} \sum_{h=1}^{H} \left|Y_{h} - \hat{Y}_{h}\right|\]
\end{enumerate}


\subsection{Short-term forecasting}

\begin{tabular}{|c|c|c|c|}
    \hline
                        & \multicolumn{3}{c|}{M4-Yearly 6}                 \\
    \cline{2-4}
    Model               & SMAPE                            & MASE  & OWA   \\
    \hline
    GPT2-small Time-LLM & 13.555                           & 3.038 & 0.797 \\
    \hline
    LLama7b Time-LLM    & 13.419                           & 3.005 & 0.789 \\
    GPT4TS              & 15.11                            & 3.565 & 0.911 \\
    TimesNet            & 15.378                           & 3.554 & 0.918 \\
    PatchTST            & 13.477                           & 3.019 & 0.792 \\
    N-HiTS              & 13.422                           & 3.056 & 0.795 \\
    ETSformer           & 13.487                           & 3.036 & 0.795 \\
    N-BEATS             & 18.009                           & 4.487 & 1.115 \\
    LightTS             & 14.247                           & 3.109 & 0.827 \\
    DLinear             & 16.965                           & 4.283 & 1.058 \\
    FEDformer           & 14.021                           & 3.036 & 0.811 \\
    Stationary          & 13.717                           & 3.078 & 0.807 \\
    Autoformer          & 13.974                           & 3.134 & 0.822 \\
    Informer            & 14.727                           & 3.418 & 0.881 \\
    Reformer            & 16.169                           & 3.800 & 0.973 \\
    \hline
\end{tabular}

\begin{enumerate}
    \item SMAPE (Symmetric Mean Absolute Percentage Error): It measures the percentage error between the forecast and the actual value without assigning a greater penalty to over- or under-prediction. The smaller the SMAPE, the better the model's performance.
          \[\text{SMAPE} = \frac{200}{H} \sum_{h=1}^{H} \frac{|Y_{h} - \hat{Y}_{h}|}{|Y_{h}| + |\hat{Y}_{h}|}\]
    \item MASE (Mean Absolute Scaled Error): It is a measure of forecast accuracy that adjusts for the variability of the time series being forecast. The smaller the MASE, the better the model's performance.
          \[\text{MASE} = \frac{1}{H} \sum_{h=1}^{H} \frac{|Y_{h} - \hat{Y}_{h}|}{\frac{1}{H-s} \sum_{j=s+1}^{H} |Y_{j} - Y_{j-s}|}\]
    \item OWA (Overall Weighted Average): It is a comprehensive metric that considers multiple evaluation metrics, often used in competitions or multi-objective optimization. The calculation method of OWA may vary depending on the application scenario.
          \[\text{OWA} = \frac{1}{2} \left[\frac{\text{SMAPE}}{\text{SMAPE}_{\text{Naive}2}} + \frac{\text{MASE}}{\text{MASE}_{\text{Naive}2}}\right]\]
\end{enumerate}


\subsection{Few-shot forecasting}

\begin{tabular}{|c|c|c|}
    \hline
                        & \multicolumn{2}{c|}{ETTh1 96 10\%}         \\
    \cline{2-3}
    Model               & MSE                                & MAE   \\
    \hline
    GPT2-small Time-LLM & 0.534                              & 0.484 \\
    \hline
    LLama7b Time-LLM    & 0.448                              & 0.460 \\
    GPT4TS              & 0.458                              & 0.456 \\
    DLinear             & 0.492                              & 0.495 \\
    PatchTST            & 0.516                              & 0.485 \\
    TimesNet            & 0.861                              & 0.628 \\
    FEDformer           & 0.512                              & 0.499 \\
    Autoformer          & 0.613                              & 0.552 \\
    Stationary          & 0.918                              & 0.639 \\
    ETSformer           & 1.112                              & 0.806 \\
    LightTS             & 1.298                              & 0.838 \\
    Informer            & 1.179                              & 0.792 \\
    Reformer            & 1.184                              & 0.790 \\
    \hline
\end{tabular}


\subsection{Zero-shot forecasting}

\begin{tabular}{|c|c|c|}
    \hline
    backbone            & \multicolumn{2}{c|}{ETTh1 $\rightarrow$ ETTh2 96}         \\
    \cline{2-3}
    model               & MSE                                               & MAE   \\
    \hline
    GPT2-small Time-LLM & 0.266                                             & 0.334 \\
    \hline
    LLama7b Time-LLM    & 0.279                                             & 0.337 \\
    LLMTime             & 0.510                                             & 0.576 \\
    GPT4TS              & 0.335                                             & 0.374 \\
    DLinear             & 0.347                                             & 0.400 \\
    PatchTST            & 0.304                                             & 0.350 \\
    TimesNet            & 0.358                                             & 0.387 \\
    Autoformer          & 0.469                                             & 0.486 \\
    \hline
\end{tabular}



\section{Analysis}

The reasons for the differences from the paper's results are:

\begin{enumerate}
    \item The main reason is that the paper used the LLAMA pre-trained model, which has significantly more model parameters and a higher model dimension than GPT2.

    \item During the training process of the paper, the ZeRO-2 optimizer under the DeepSpeed framework was utilized. This is a technology optimized and accelerated for large-scale training. In this process, gradient accumulation was used to simulate larger batch sizes, but it also introduced randomness. Secondly, the paper adopted mixed-precision training, which can significantly reduce memory usage and speed up training. However, using lower precision in floating-point representation also introduces additional numerical computation errors, which is another source of result randomness. Our training process was conducted on a single GPU, which also causes some deviation in the results.

    \item Different CUDA versions can also affect the model results.
\end{enumerate}


\end{document}
